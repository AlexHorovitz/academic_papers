\documentclass{article}
\usepackage{amsmath}
\usepackage{amssymb}
\usepackage{footnote}
\usepackage[style=apa]{biblatex}
\addbibresource{still_point.bib}

\title{The Still Point of a Turning World: Originary Temporality and the Paradox of Heidegger's Formal Indication}
\author{Alex Horovitz}
\date{April 2025}

\begin{document}
\setlength{\parskip}{0.75em}

\maketitle

\section*{Introduction}

Martin Heidegger's project in \textit{Being and Time} seeks to rethink the meaning of being itself by tracing it back to its existential roots in human temporality\footnote{Heidegger, M. (1962). \textit{Being and Time} (J. Macquarrie \& E. Robinson, Trans.). Harper \& Row. (Original work published 1927)}. Heidegger believes that we cannot properly understand what it means \textit{to be} if we conceive of time merely as a sequence of measurable moments (clock time). Rather, he proposes that being is intelligible only through a deeper, more originary form of temporality: a structure of existence in which the future, the past, and the present are interwoven through our lived experiences. Time then, for Heidegger, is not an external dimension but the very fabric that makes any understanding of being possible.

A major interpretive aid to Heidegger's thought is provided by William Blattner, who clarifies that Dasein --- Heidegger's term for the human being --- is structured by three inseparable temporal dimensions: projection into future possibilities, thrownness into a factical past that matters, and absorption in a meaningful present\footnote{Blattner, W. (2007). Temporality. In H. L. Dreyfus \& M. A. Wrathall (Eds.), \textit{A Companion to Heidegger} (pp. 311--324). Blackwell Publishing.}. Heidegger calls this structure \textit{originary temporality}, emphasizing that it is not a sequence of events but an ecstatic unity. Our self-understanding, our engagements with the world, and even our experience of others are conditioned by this prior temporal horizon.

Through his efforts, Heidegger is also seeking to avoid the pitfalls of traditional metaphysics, which he accuses of turning fluid existential phenomena into rigid, objectified categories. To this end, he introduces the method of \textit{formal indication}, a way of pointing toward existential structures without treating them as theoretical constructs\footnote{Dahlstrom, D. O. (1994). Heidegger's Method: Philosophical Concepts as Formal Indications. \textit{The Review of Metaphysics, 47}(4), 775--795.}. Formal indications are supposed to guide inquiry without solidifying into static concepts. Yet Heidegger faces a critical difficulty: any attempt to describe or explain formal indications requires language, and language inevitably tends toward fixation and objectification; even in our private internal thoughts.

By this paper's account, despite Heidegger's best efforts, his method of formal indication ultimately succumbs to a self-referential paradox. Using formal predicate logic, I will show that the conditions Heidegger sets for formal indication contradict the necessity of articulating it in discourse. Furthermore, I will argue that even if formal indications are performative rather than propositional, they still rely on minimal conceptual structures that deeply undermine any goal around resisting objectification. While Heidegger's project may, for some, reveal profound insights into human existence, it ultimately forfeits its ability to serve as a viable method for repeatable inquiry.

\section*{Heidegger's Critique of Traditional Metaphysics}

Heidegger's \textit{Being and Time} begins with a radical challenge to the basic assumptions of traditional Western metaphysics. From the time of Plato and Aristotle onward, metaphysics has been dominated by the quest to grasp beings as present-at-hand: stable, discrete objects whose properties can be analyzed and catalogued. In this tradition, "Being" itself tends to recede behind the beings it makes possible. Heidegger famously claims that the history of philosophy has forgotten the question of Being precisely because it has prioritized beings \textit{as} objects, encountered primarily through categories of presence, identity, and substance\footnote{Heidegger, M. (1962). \textit{Being and Time} (J. Macquarrie \& E. Robinson, Trans.). Harper \& Row. (Original work published 1927).}.

Against this backdrop, Heidegger proposes that Being must be rethought through the lived experience of Dasein, the entity for whom Being is an issue. Rather than understanding Being in terms of static properties or eternal forms, Heidegger emphasizes that Being is disclosed dynamically through time. Dasein's existence is characterized not by mere presence but by care, projection, thrownness, and involvement --- structures that only make sense within a temporal horizon. Thus, metaphysicians (indeed metaphysics as a practice) are in error when they treat Being as simply "there" like an object, rather than as something that emerges, withdraws, and becomes intelligible \underline{\textbf{only}} in relation to Dasein's temporal existence.

To avoid replicating the objectifying habits of traditional metaphysics, Heidegger introduces the method of \textit{formal indication}. Rather than define or categorize existential structures, formal indications are meant to gesture toward them without reifying them into fixed entities. The goal is to maintain openness and allow Being to disclose itself on its own terms, rather than forcing it into preconceived conceptual boxes\footnote{Dahlstrom, D. O. (1994). Heidegger's Method: Philosophical Concepts as Formal Indications. \textit{The Review of Metaphysics, 47}(4), 775--795.}.

Yet this methodological ambition will raise serious tensions, as later sections will show. If Being can only be disclosed through Dasein's lived temporality, and if articulation itself tends toward objectification, then even Heidegger's carefully provisional method risks betraying its own purpose. This tension forms the basis of the deeper paradox at the heart of \textit{Being and Time}, one that ultimately challenges the viability of formal indication as a method for repeatable inquiry.

\end{document}
