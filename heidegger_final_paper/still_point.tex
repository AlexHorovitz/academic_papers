\documentclass{article}
\usepackage{amsmath}
\usepackage{amssymb}
\usepackage{footnote}
\usepackage[style=apa]{biblatex}
\addbibresource{still_point.bib}

\title{The Still Point of a Turning World: Originary Temporality and the Paradox of Heidegger's Formal Indication}
\author{Alex Horovitz \\ ahorovitz@sfsu.edu}
\date{May 2025}

\begin{document}
\setlength{\parskip}{0.75em}

\maketitle

\section*{Introduction}

Martin Heidegger's project in \textit{Being and Time} seeks to rethink the meaning of being itself by tracing it back to its existential roots in human temporality\footnote{Heidegger, M. (1962). \textit{Being and Time} (J. Macquarrie \& E. Robinson, Trans.). Harper \& Row. (Original work published 1927)}. Heidegger believes that we cannot properly understand what it means \textit{to be} if we conceive of time merely as a sequence of measurable moments (clock time). Rather, he proposes that being is intelligible only through a deeper, more originary form of temporality: a structure of existence in which the future, the past, and the present are interwoven through our lived experiences. Time then, for Heidegger, is not an external dimension but the very fabric that makes any understanding of being possible.

A major interpretive aid to Heidegger's thought is provided by William Blattner, who clarifies that Dasein --- Heidegger's term for the human being --- is structured by three inseparable temporal dimensions: projection into future possibilities, thrownness into a factical past that matters, and absorption in a meaningful present\footnote{Blattner, W. (2007). Temporality. In H. L. Dreyfus \& M. A. Wrathall (Eds.), \textit{A Companion to Heidegger} (pp. 311--324). Blackwell Publishing.}. Heidegger calls this structure \textit{originary temporality}, emphasizing that it is not a sequence of events but an ecstatic unity. Our self-understanding, our engagements with the world, and even our experience of others are conditioned by this prior temporal horizon.

Through his efforts, Heidegger is also seeking to avoid the pitfalls of traditional metaphysics, which he accuses of turning fluid existential phenomena into rigid, objectified categories. To this end, he introduces the method of \textit{formal indication}, a way of pointing toward existential structures without treating them as theoretical constructs\footnote{Dahlstrom, D. O. (1994). Heidegger's Method: Philosophical Concepts as Formal Indications. \textit{The Review of Metaphysics, 47}(4), 775--795.}. Formal indications are supposed to guide inquiry without solidifying into static concepts. Yet Heidegger faces a critical difficulty: any attempt to describe or explain formal indications requires language, and language inevitably tends toward fixation and objectification; even in our private internal thoughts.

By this paper's account, despite Heidegger's best efforts, his method of formal indication ultimately succumbs to a self-referential paradox. Using formal predicate logic, I will show that the conditions Heidegger sets for formal indication contradict the necessity of articulating it in discourse. Furthermore, I will argue that even if formal indications are performative rather than propositional, they still rely on minimal conceptual structures that deeply undermine any goal around resisting objectification. While Heidegger's project may, for some, reveal profound insights into human existence, it ultimately forfeits its ability to serve as a viable method for repeatable inquiry.

\section*{Heidegger's Critique of Traditional Metaphysics}

Heidegger's \textit{Being and Time} begins with a radical challenge to the basic assumptions of traditional Western metaphysics. From the time of Plato and Aristotle onward, metaphysics has been dominated by the quest to grasp beings as present-at-hand: stable, discrete objects whose properties can be analyzed and catalogued. In this tradition, "Being" itself tends to recede behind the beings it makes possible. Heidegger famously claims that the history of philosophy has forgotten the question of Being precisely because it has prioritized beings \textit{as} objects, encountered primarily through categories of presence, identity, and substance\footnote{Heidegger, M. (1962). \textit{Being and Time} (J. Macquarrie \& E. Robinson, Trans.). Harper \& Row. (Original work published 1927).}.

Against this backdrop, Heidegger proposes that Being must be rethought through the lived experience of Dasein, the entity for whom Being is an issue. Rather than understanding Being in terms of static properties or eternal forms, Heidegger emphasizes that Being is disclosed dynamically through time. Dasein's existence is characterized not by mere presence but by care, projection, thrownness, and involvement --- structures that only make sense within a temporal horizon. Thus, metaphysicians (indeed metaphysics as a practice) are in error when they treat Being as simply "there" like an object, rather than as something that emerges, withdraws, and becomes intelligible \underline{\textbf{only}} in relation to Dasein's temporal existence.

To avoid replicating the objectifying habits of traditional metaphysics, Heidegger introduces the method of \textit{formal indication}. Rather than define or categorize existential structures, formal indications are meant to gesture toward them without reifying them into fixed entities. The goal is to maintain openness and allow Being to disclose itself on its own terms, rather than forcing it into preconceived conceptual boxes\footnote{Dahlstrom, D. O. (1994). Heidegger's Method: Philosophical Concepts as Formal Indications. \textit{The Review of Metaphysics, 47}(4), 775--795.}.

Yet this methodological ambition will raise serious tensions, as later sections will show. If Being can only be disclosed through Dasein's lived temporality, and if articulation itself tends toward objectification, then even Heidegger's carefully provisional method risks betraying its own purpose. This tension forms the basis of the deeper paradox at the heart of \textit{Being and Time}, one that ultimately challenges the viability of formal indication as a method for repeatable inquiry.

\section*{What Is Originary Temporality? \\ (Following Blattner)}

Central to Heidegger's rethinking of Being is the concept of \textit{originary temporality}. Unlike the traditional view of time as a series of measurable "nows" moving along a linear continuum, originary temporality describes the fundamental structure through which Dasein experiences and discloses Being. Time, in this existential sense, is not something external to human existence; it is the very fabric/horizon that makes existence and understanding possible\footnote{Heidegger, M. (1962). \textit{Being and Time} (J. Macquarrie \& E. Robinson, Trans.). Harper \& Row. (Original work published 1927).}.

William Blattner provides a particularly lucid interpretation of Heidegger's complex account of temporality. According to Blattner, Dasein's existence is structured by three interrelated dimensions: projection, thrownness, and falling\footnote{Blattner, W. (2007). Temporality. In H. L. Dreyfus \& M. A. Wrathall (Eds.), \textit{A Companion to Heidegger} (pp. 311--324). Blackwell Publishing.}. Projection refers to Dasein's orientation toward future possibilities --- its constant movement ahead of itself in pursuit of meaning and fulfillment. Thrownness captures the fact that Dasein always already finds itself embedded in a world it did not choose, shaped by conditions, moods, and histories that matter to it. Falling describes Dasein's tendency to lose itself in the distractions and norms of everyday social existence.

These three dimensions do not occur sequentially, as if Dasein first is thrown, then projects, and then falls. Instead, they form an \textit{ecstatic unity} --- a standing out into time as a whole. Dasein is always already ahead of itself (future), grounded in what has mattered (past), and absorbed in its current engagements (present). Time is thus not a container within which events happen, but a dynamic, integrated openness that structures Dasein's very being. This view radically departs from Cartesian or Newtonian conceptions of time as a neutral backdrop to existence.

In Heidegger's view, originary temporality is the condition for any intelligibility at all. Without this temporal horizon, beings would not appear as meaningful or available for concernful engagement. Tools would not show up as "to be used," projects would not beckon as "to be realized," and even other human beings would not show up as "to be encountered." Temporality underwrites all understanding, interpretation, and disclosure. Heidegger thus reverses the traditional metaphysical relation: instead of time being a feature of beings, beings show up against the background of temporality.

Blattner emphasizes that originary temporality remains largely invisible in everyday life. We usually encounter beings and deal with situations without noticing the temporal structures that make them possible. It is only when disruptions occur --- when a tool breaks, a project collapses, or anxiety reveals the null basis of our possibilities --- that we become explicitly aware of the temporal dynamics at play\footnote{Blattner, W. (2007). Temporality. In H. L. Dreyfus \& M. A. Wrathall (Eds.), \textit{A Companion to Heidegger} (pp. 311--324). Blackwell Publishing.}.

Understanding originary temporality is crucial for appreciating why Heidegger rejects traditional metaphysical methods. If Being is disclosed only through the shifting, interwoven dimensions of human temporal existence, then any method that tries to define Being once and for all --- outside of time --- is doomed to failure. Heidegger's reliance on formal indication must be seen against this backdrop: an attempt to hint at structures that can never be finally captured, precisely because they are rooted in the fluid, dynamic process of temporal disclosure.

\section*{The Challenge of Articulating  \\ Without Objectifying}

Heidegger is acutely aware that any philosophical effort to speak about Being risks falling into the very objectification he seeks to avoid. Traditional metaphysics, by naming and categorizing beings, reifies them---turning dynamic phenomena into static entities. Heidegger introduces the method of \textit{formal indication} to resist this tendency. Rather than presenting existential structures as fully articulated concepts, formal indications aim merely to gesture toward phenomena, preserving their openness and fluidity\footnote{Dahlstrom, D. O. (1994). Heidegger's Method: Philosophical Concepts as Formal Indications. \textit{The Review of Metaphysics, 47}(4), 775--795.}.

The method of formal indication thus intends to perform a fundamentally different role than traditional philosophical definition. A formal indication does not offer a closed, content-laden concept but rather points the inquirer toward an experience or structure that must be encountered and interpreted through lived existence. Heidegger hopes that by keeping indications formally "empty" of determinate content, he can avoid reifying the phenomena he aims to disclose.

However, a significant tension arises here. Even if a formal indication refrains from propositional definition, it must still be articulated---spoken, written, communicated---in language. Language, by its very nature, imposes structure. To articulate is to pick out, delimit, and stabilize meaning within a system of signs. Even if Heidegger intends his indications to function performatively rather than descriptively, the very act of articulation inevitably carries with it the danger of fixation and conceptual closure.

This problem is not merely external or practical; it strikes at the heart of Heidegger's method. If Being is disclosed only through the dynamic, temporal structures of existence, then even a formally indicated pointing risks freezing what should remain fluid. The performative ambition to "show" rather than "say" cannot fully evade the sedimentation of meaning that occurs whenever we use language to share an experience. In other words, even gestures toward openness become susceptible to reification once they are articulated.

Some defenders of Heidegger argue that this risk is precisely why he emphasizes \textit{repeated engagement} with formal indications---why phenomenological inquiry must be ongoing, circular, and revisable. However, this strategy does not eliminate the underlying issue. If the method depends on endless reinterpretation to resist closure, it raises questions about whether any stable understanding or critical discourse can be achieved at all. As we will see in the next section, this tension can be formalized into a paradox that challenges the coherence of formal indication as a method for repeatable inquiry.


\end{document}
