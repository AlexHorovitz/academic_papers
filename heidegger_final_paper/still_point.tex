\documentclass{article}
\usepackage{amsmath}
\usepackage{amssymb}
\usepackage[style=apa, backend=biber]{biblatex}
\setlength{\bibitemsep}{1.5\baselineskip}

\DeclareLanguageMapping{american}{american-apa}
\addbibresource{still_point.bib}

\title{The Still Point of a Turning World: Originary Temporality and the Paradox of Heidegger's Formal Indication}
\author{Alex Horovitz \\ \small ahorovitz@mail.sfsu.edu}
\date{May 2025}

\begin{document}
\setlength{\parskip}{0.75em}

\maketitle

\section*{Introduction}

Martin Heidegger's project in \textit{Being and Time} seeks to rethink the meaning of being itself by tracing it back to its existential roots in human temporality \parencite[p.~34]{heidegger1962}. Heidegger believes that we cannot properly understand what it means \textit{to be} if we conceive of time merely as a sequence of measurable moments (clock time). Rather, he proposes that being is intelligible only through a deeper, more originary form of temporality: a structure of existence in which the future, the past, and the present are interwoven through our lived experiences. Time then, for Heidegger, is not an external dimension but the very fabric that makes any understanding of being possible.

A major interpretive aid to Heidegger's thought is provided by William Blattner, who clarifies that Dasein --- Heidegger's term for the human being --- is structured by three inseparable temporal dimensions: projection into future possibilities, thrownness into a factical past that matters, and absorption in a meaningful present \parencite[pp.~311--312]{blattner2007}. Heidegger calls this structure \textit{originary temporality}, emphasizing that it is not a sequence of events but an ecstatic unity. Our self-understanding, our engagements with the world, and even our experience of others are conditioned by this prior temporal horizon.

Through his efforts, Heidegger is also seeking to avoid the pitfalls of traditional metaphysics, which he accuses of turning fluid existential phenomena into rigid, objectified categories. To this end, he introduces the method of \textit{formal indication}, a way of pointing toward existential structures without treating them as theoretical constructs \parencite[p.~782]{dahlstrom1994}. Formal indications are supposed to guide inquiry without solidifying into static concepts. Yet Heidegger faces a critical difficulty: any attempt to describe or explain formal indications requires language, and language inevitably tends toward fixation and objectification; even in our private internal thoughts.

By this paper's account, despite Heidegger's best efforts, his method of formal indication ultimately succumbs to a self-referential paradox. Using formal predicate logic, I will show that the conditions Heidegger sets for formal indication contradict the necessity of articulating it in discourse. Furthermore, I will argue that even if formal indications are performative rather than propositional, they still rely on minimal conceptual structures that deeply undermine any goal around resisting objectification. While Heidegger's project may, for some, reveal profound insights into human existence, it ultimately forfeits its ability to serve as a viable method for repeatable inquiry.

\section*{Heidegger's Critique of Traditional Metaphysics}

Heidegger's \textit{Being and Time} begins with a radical challenge to the basic assumptions of traditional Western metaphysics. From the time of Plato and Aristotle onward, metaphysics has been dominated by the quest to grasp beings as present-at-hand: stable, discrete objects whose properties can be analyzed and catalogued. In this tradition, "Being" itself tends to recede behind the beings it makes possible. Heidegger famously claims that the history of philosophy has forgotten the question of Being precisely because it has prioritized beings \textit{as} objects, encountered primarily through categories of presence, identity, and substance \parencite[p.~29]{heidegger1962}.

Against this backdrop, Heidegger proposes that Being must be rethought through the lived experience of Dasein, the entity for whom Being is an issue. Rather than understanding Being in terms of static properties or eternal forms, Heidegger emphasizes that Being is disclosed dynamically through time. Dasein's existence is characterized not by mere presence but by care, projection, thrownness, and involvement --- structures that only make sense within a temporal horizon. Thus, metaphysicians (indeed metaphysics as a practice) are in error when they treat Being as simply "there" like an object, rather than as something that emerges, withdraws, and becomes intelligible \underline{\textbf{only}} in relation to Dasein's temporal existence.

To avoid replicating the objectifying habits of traditional metaphysics, Heidegger introduces the method of \textit{formal indication}. Heidegger's concept of formal indication functions both to prohibit premature objectification by guiding inquiry without specifying its object ("referring-prohibitive") and to transform the philosopher through a reversal of habitual thought patterns ("reversing-transforming"), distinguishing it from logical or thematic concepts by requiring the thinker’s active engagement for understanding. \parencite[pp.~783--784]{dahlstrom1994}. The goal is to maintain openness and allow Being to disclose itself on its own terms, rather than forcing it into preconceived conceptual boxes.

Yet this methodological ambition will raise serious tensions, as later sections will show. If Being can only be disclosed through Dasein's lived temporality, and if articulation itself tends toward objectification, then even Heidegger's carefully provisional method risks betraying its own purpose. This tension forms the basis of the deeper paradox at the heart of \textit{Being and Time}, one that ultimately challenges the viability of formal indication as a method for repeatable inquiry.

\section*{What Is Originary Temporality? \\ (Following Blattner)}

Central to Heidegger's rethinking of Being is the concept of \textit{originary temporality}. Unlike the traditional view of time as a series of measurable "nows" moving along a linear continuum, originary temporality describes the fundamental structure through which Dasein experiences and discloses Being. Time, in this existential sense, is not something external to human existence; it is the very fabric/horizon that makes existence and understanding possible \parencite[p.~39]{heidegger1962}.

William Blattner provides a particularly lucid interpretation of Heidegger's complex account of temporality. According to Blattner, Dasein's existence is structured by three interrelated dimensions: projection, thrownness, and falling \parencite[pp.~311--312]{blattner2007}. Projection refers to Dasein's orientation toward future possibilities --- its constant movement ahead of itself in pursuit of meaning and fulfillment. Thrownness captures the fact that Dasein always already finds itself embedded in a world it did not choose, shaped by conditions, moods, and histories that matter to it. Falling describes Dasein's tendency to lose itself in the distractions and norms of everyday social existence.

These three dimensions do not occur sequentially, as if Dasein first is thrown, then projects, and then falls. Instead, they form an \textit{ecstatic unity} --- a standing out into time as a whole. Dasein is always already ahead of itself (future), grounded in what has mattered (past), and absorbed in its current engagements (present). Time is thus not a container within which events happen, but a dynamic, integrated openness that structures Dasein's very being. This view radically departs from Cartesian or Newtonian conceptions of time as a neutral backdrop to existence.

In Heidegger's view, originary temporality is the condition for any intelligibility at all. Without this temporal horizon, beings would not appear as meaningful or available for concernful engagement. Tools would not show up as "to be used," projects would not beckon as "to be realized," and even other human beings would not show up as "to be encountered." Temporality underwrites all understanding, interpretation, and disclosure \parencite[pp.~645--650]{heidegger1962}. Heidegger thus reverses the traditional metaphysical relation: instead of time being a feature of beings, beings show up against the background of temporality.

Blattner explains that in everyday life, Dasein is absorbed in its practical projects and engagements, encountering things like tools and tasks in terms of what they enable, rather than thematically focusing on its own being. Because of this, the temporal structure underlying these engagements typically remains concealed—becoming explicit only when habitual activity is disrupted and the world no longer functions smoothly. \parencite[p.~314]{blattner2007}.

Understanding originary temporality is crucial for appreciating why Heidegger rejects traditional metaphysical methods. If Being is disclosed only through the shifting, interwoven dimensions of human temporal existence, then any method that tries to define Being once and for all --- outside of time --- is doomed to failure. Heidegger's reliance on formal indication must be seen against this backdrop: an attempt to hint at structures that can never be finally captured, precisely because they are rooted in the fluid, dynamic process of temporal disclosure.

\section*{The Challenge of Articulating \\ Without Objectifying}

Heidegger is acutely aware that any philosophical attempt to speak about Being risks falling into the very objectification he seeks to avoid. Traditional metaphysics, by naming and categorizing beings, reifies them—turning dynamic phenomena into static entities. In response, Heidegger introduces the method of \textit{formal indication}, which avoids defining existential structures directly. Instead, it gestures toward phenomena, preserving their fluidity by refusing to reduce them to conceptual objects \parencite[pp.~781–784]{dahlstrom1994}. <-

Unlike traditional definitions, formal indications are meant to function as open-ended prompts or “guides” rather than statements of fact. Heidegger emphasizes that a formal indication does not supply content in the usual sense; rather, it points toward something that must be disclosed through lived, temporal engagement. This gesture is supposed to help the inquirer remain within the sphere of existential experience without converting it into fixed theoretical content \parencite[pp.~781–783]{dahlstrom1994}.

However, even if formal indications avoid propositional content, they must still be articulated—spoken, written, or otherwise communicated. And herein lies the paradox: language imposes structure. To articulate is to select, delimit, and stabilize meaning. Even a gesture becomes interpretable only within a shared framework of signs. Thus, the performative aim of formal indication—to “show” rather than “say”—cannot fully escape the gravitational pull of conceptual fixation.

This is not simply a practical limitation, but a methodological contradiction. If Being is disclosed only through Dasein’s temporal existence, then even an effort to indicate formally cannot fully resist reification. The structure of language itself tends toward objectification—even when used cautiously. Dahlstrom acknowledges this danger, noting that even formal indications risk being mistaken for conceptual assertions by virtue of their discursive form \parencite[p.~783]{dahlstrom1994}.

Some defenders of Heidegger argue that the solution lies in repetition: that formal indications are meant to be reengaged, reinterpreted, and reinhabited through ongoing phenomenological inquiry. But this strategy only defers the problem. If the method relies on endless return to avoid closure, then it forfeits the clarity, stability, and repeatability expected of any method that aspires to produce shared knowledge. Let us now turn our attention to show how this conceptual tension leads to a formal paradox.

\section*{Formalization of the Paradox}

The paradox underlying Heidegger’s method of formal indication can be made precise using predicate logic. The aim is not to caricature his phenomenological approach as strictly logical, but to illustrate that his methodological commitments lead to a contradiction when examined under the minimal structural conditions of language and discourse.

To begin, let us define four predicates:
\begin{itemize}
  \item $F(x)$: $x$ is a formal indication.
  \item $P(x)$: $x$ is provisional and open-ended.
  \item $A(x)$: $x$ is articulated in language.
  \item $O(x)$: $x$ is objectified (conceptually fixed).
\end{itemize}

We now consider four plausible assumptions grounded in Heidegger’s method and Dahlstrom’s analysis:

\begin{enumerate}
  \item $\forall x (F(x) \rightarrow P(x))$ \hfill \textit{(Formal indications must remain provisional)} \parencite[pp.~781–782]{dahlstrom1994}.
  \item $\forall x (F(x) \rightarrow A(x))$ \hfill \textit{(Formal indications must be articulated to be communicated)} \parencite[p.~781]{dahlstrom1994}.
  \item $\forall x (A(x) \rightarrow O(x))$ \hfill \textit{(Articulation in language entails objectification)} \parencite[p.~783]{dahlstrom1994}.
  \item $\forall x (O(x) \rightarrow \neg P(x))$ \hfill \textit{(Objectification undermines provisionality)}.
\end{enumerate}

From (1), (2), (3), and (4), we can derive a contradiction:
\[
F(x) \rightarrow A(x) \rightarrow O(x) \rightarrow \neg P(x)
\]
But by (1), $F(x) \rightarrow P(x)$. Thus, we derive both $P(x)$ and $\neg P(x)$ from the same premise, $F(x)$—a contradiction.

This logical This tension is not merely an external or semantic inconvenience, but arises from the very structure of philosophizing itself. As Dahlstrom explains, philosophizing is a mode of being-in-the-world that seeks to determine its own conditions of possibility. Because our being is disclosed pre-thematically—through mood, understanding, public involvement, and language—philosophical concepts must function as formal indications: they must gesture toward what is always already there, yet typically concealed \parencite[pp.~789-790]{dahlstrom1994}. These concepts, such as “care,” “world,” or “existence,” point to tasks that philosophy must perform, rather than serve as fixed definitions. Yet this very need for expression—this act of “retrieving” and “reiterating” what is forgotten—exposes formal indications to the structural tension they aim to avoid. As soon as they are articulated, they begin to function like the thematic concepts they resist. Heidegger’s method, then, is not immune to the performative contradiction that haunts any effort to speak the unspeakable without reifying it.

Thus, while Heidegger’s method aims to gesture without conceptual closure, it cannot avoid the basic problem that gestures too must be made within the medium of discourse. If every indication necessarily becomes part of a shared interpretive structure, then the goal of preserving openness collapses. Formal indication, by its own conditions, defeats itself.

\section*{The Limits of Performative Methods \\ for Repeatable Inquiry}

While Heidegger's method of formal indication was designed to resist objectification, it also resists the basic conditions required for knowledge to be transferable and subject to critical scrutiny. The ambition to “show” without saying—through gestures rather than definitions—may safeguard existential openness, but it forfeits one of the fundamental virtues of philosophical and scientific inquiry: repeatability. A method that cannot be applied consistently, shared across interpreters, or falsified, ceases to function as a method in any epistemically meaningful sense.

Philosophers of science such as Karl Popper have long insisted that the test of any theoretical claim lies in its falsifiability—its vulnerability to disproof \parencite[pp.~39–42]{popper2002}. Heidegger's performative disclosures fail this test by design. Because they cannot be reproduced in controlled, conceptually stable terms, they elude not only empirical verification but also systematic critique. Instead, they become more akin to poetic insights or spiritual revelations—valuable perhaps in a personal or therapeutic context, but inadequate as a basis for shared inquiry. As Daniel Dennett once wryly noted, “... and the philosophers, as we all know, just take in each other’s laundry, warning about confusions they themselves have created, in an arena bereft of both data and empirically testable theories” \parencite[p.~255]{dennett1991}.

Even within phenomenology, a method must allow for *intersubjective validity*. Maurice Merleau-Ponty, while sympathetic to Heidegger’s project, emphasizes that phenomenological description must be “communicable and checkable” by others, even if it is not strictly scientific \parencite[p.~xv]{merleauponty1962}. Heidegger’s notion of formal indication seems to violate even this weaker standard. If each encounter with Being must be grasped anew and cannot be framed in a stable conceptual structure, then no community of inquiry—whether philosophical or scientific—can build on such a method.

The result is a deeply paradoxical stance: Heidegger's method draws our attention to the preconceptual structures of meaning-making, but it also forecloses the possibility of developing cumulative knowledge. One may say that it offers a critique of metaphysical closure without offering a viable alternative. It tells us what not to do, but not how to proceed constructively without falling back into the very structures it aims to dismantle. In this way, Heidegger’s method ultimately resembles what Thomas Kuhn described as a “pre-paradigmatic” approach—rich in insight but lacking the shared models, rules, and standards that would enable a coherent research tradition to emerge or progress through cumulative inquiry. \parencite[pp.~10-11]{kuhn1996}.

Thus, even if we grant Heidegger the legitimacy of a non-traditional mode of thinking, we must acknowledge that it comes at the cost of rigor, repeatability, and epistemic accountability. What remains is a powerful existential critique, but not a philosophical method in the sense that supports collaborative, disciplined pursuit of truth.

\section*{Counterarguments and Responses}

In defense of Heidegger, one might argue that the criticism advanced so far misunderstands the nature of his project. Heidegger is not offering a method in the analytic or scientific sense but proposing a radical reorientation of thought—an attempt to rethink the very grounds on which inquiry proceeds. In this light, to demand that his method be repeatable or falsifiable may be to impose the wrong evaluative framework altogether. As Hubert Dreyfus notes, Heidegger aims not to provide criteria for scientific verification but to uncover the “preontological or pretheoretical" intelligibility that makes any kind of sense-making possible in the first place \parencite[pp.~19--23]{dreyfus1991}.

Moreover, Heidegger is explicit in his rejection of traditional epistemology, particularly the Cartesian subject-object schema. His focus is on disclosure, not representation; understanding, not justification. From this perspective, the role of formal indication is not to define or test a hypothesis, but to open a space of meaning where Being may emerge. Critics who demand epistemic repeatability may thus be missing the point: Heidegger is not seeking to produce knowledge in the modern scientific sense but to ground it in existential conditions.

This position is echoed by later Heideggerians and phenomenologists. For example, James Hatley defends Heidegger’s language as intentionally “poietic”—that is, not merely poetic but generative of disclosure through evocative speech acts \parencite[p.~108]{hatley2000}. In this view, the apparent obscurity or evasiveness of Heidegger's writing is not a flaw but a necessity: it resists the reduction of Being to presence and of meaning to reference. Philosophical rigor, then, would be redefined not as analytic clarity but as existential fidelity.

Nonetheless, this line of defense ultimately concedes the central critique: that Heidegger's approach resists the very conditions that make philosophical discourse cumulative, collaborative, and contestable. If every act of interpretation must begin anew—without stable reference points or shared concepts—then each engagement becomes self-contained. As Rorty notes, “to say that Being is disclosed, but never adequately captured, is to invite a conversation in which each speaker speaks only to herself” \parencite[p.~27]{rorty1991}. At best, this results in a fragmented mode of inquiry; at worst, it becomes a refusal of discourse altogether.

More importantly, even if Heidegger is not offering a method in the narrow sense, he still claims to be disclosing something fundamental about the human condition. But to disclose something—rather than merely evoke or provoke—requires that others be able to recognize, understand, and respond to that disclosure. Otherwise, the claim loses intersubjective traction and collapses into hermeneutic solipsism. As Charles Taylor argues, even interpretive philosophy must “give reasons that others can follow, contest, or extend” \parencite[p.~21]{taylor1985}. Without this, the project may retain existential depth, but not epistemic credibility.

This levaves us in a state where the strongest defense of Heidegger's performative method grants its evocative power but also confirms its limitation: it cannot function as a method for producing shareable, repeatable, and correctable knowledge. In this sense, Heidegger's critique of metaphysics becomes its own metaphysical stance—resistant to falsification, insulated from counterargument, and ultimately self-sealing. While it may continue to inspire, it cannot guide inquiry in the disciplined sense that philosophy, at its best, aspires to.

\section*{Conclusion: Potential Pitfalls of Philosophical Romanticism}

Heidegger’s project in \textit{Being and Time} seeks to break free from the rigid conceptual frameworks of traditional metaphysics. Through the use of formal indication and the concept of originary temporality, he attempts to recast the nature of Being as something disclosed through lived, temporal experience rather than captured by fixed propositions. This bold move has inspired generations of philosophers, particularly in the continental tradition, to rethink the foundations of meaning, subjectivity, and existence.
\newpage
Yet, as this paper has asserted, the very mechanisms Heidegger employs to preserve existential openness—especially the method of formal indication—are themselves caught in a paradox. When articulated in language, formal indications inevitably take on the structure of the very conceptual fixity they aim to avoid. The result is a self-undermining tension: Heidegger’s method resists reification, yet must operate through the mediums of thought and speech that reify by default. Once we attempt to share or communicate the gesture, it becomes something more than a gesture—it becomes a proposition, a reference, a claim.

This methodological impasse reveals a broader philosophical tendency that might be called romanticism: the belief that genuine insight lies beyond articulation, that truth must be gestured toward rather than captured, and that meaning emerges only through singular, irreducible acts of disclosure. While this attitude can generate powerful critiques of scientism and reductive thinking, it also risks withdrawing from the shared space of reasoned inquiry. When philosophy privileges ineffability over argument, it can become a kind of intellectual monologue—provocative but unaccountable.

If philosophy is to remain a discipline grounded in both rigor and responsibility, it must balance existential depth with discursive clarity. Even when we challenge prevailing conceptual schemes, we must offer others a way to engage, interpret, and critique those challenges. Without such engagement, philosophy risks lapsing into mysticism or aestheticism—forms of expression that may move us, but which lack the capacity to build cumulative insight. As Paul Ricoeur once warned, “hermeneutics must avoid the twin dangers of positivism and romanticism” \parencite[p.~23]{ricoeur1976}. Heidegger, in his effort to escape the former, may have fallen too far into the latter.

To be clear, this is not a dismissal of Heidegger’s contribution. Rather, it is a call to recover what is vital in his work—his sensitivity to lived meaning and temporality—while resisting the temptation to abandon the discursive virtues that make philosophy not just an act of thinking, but a public and repeatable practice of inquiry.

\newpage
\printbibliography

\end{document}
