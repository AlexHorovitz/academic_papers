\documentclass[11pt]{article}
\usepackage[margin=1in]{geometry}
\usepackage{amsmath, amssymb}
\usepackage{setspace}
\usepackage{parskip}
\usepackage{hyperref}
\hypersetup{colorlinks=true, linkcolor=blue, urlcolor=blue, citecolor=blue}

\title{Response to Matthew Ibrahim on Wolin's \textit{"Earth and Soil"}}
\author{}
\date{}

\begin{document}
\maketitle

Matthew,

Your paper on Richard Wolin's reading of Heidegger in \textit{Heidegger in Ruins} is cogent, ambitious, and intellectually generous. You succeed in parsing the two primary threads of Wolin's argument—the historical-ideological indictment and the philosophical critique—and your response is refreshingly honest in its attempt to wrestle with the tension between Heidegger's intellectual brilliance and his moral-political failures. Your ability to move between close textual reading and broader philosophical reflection is commendable. In particular, your use of Foucault to frame the stakes of Enlightenment critique offers a sharp counterpoint to Wolin's Enlightenment-humanist posture.

That said, there are two areas where your argument could be deepened. 

Firstly, you suggest that Wolin elides the distinction between Heidegger's contingent ideological entanglements and the broader legitimacy of his conceptual project. But could it not be the case that Heidegger's philosophical concepts are not merely incidentally entangled with reactionary politics, but structurally predisposed toward them? That is, might the very grammar of \textit{Seinsgeschichte}, \textit{Ereignis}, and \textit{Bodenst\"andigkeit} invite a mythic and anti-modern sensibility, regardless of authorial intent? You challenge Wolin to justify the rejection of Heidegger's critique of reason, but you might ask yourself whether that critique itself leaves room for a normative defense of pluralism and difference.

Secondly, while you defend the philosophical autonomy of Heidegger's categories, you concede that his vision was one of "ontological rootedness" that proved, in practice, exclusionary and dangerous. You stop short, however, of asking what kind of political vision can emerge from an ontology that denies contingency, mobility, and rootlessness as authentic forms of life. If Heidegger's framework cannot affirm the nomadic or the diasporic as existentially valid, can it truly speak to the conditions of modern being-in-the-world?

In light of these points, I leave you with a question: \textbf{What kind of thinking is possible that acknowledges the tragedy of rootlessness without mythologizing the soil—and can Heidegger help us think such a path, or must we go beyond him?}

Warm regards,

Alex Horovitz

\end{document}
